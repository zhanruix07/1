\section{Results and Discussion}
\label{sec:results}
You should present all relevant information in this section objectively. There is no need to include calculations, that can be shown in appendix, (for instance as such: see appendix \ref{app:first_appendix}), but be sure that all measurements and end results are in the report. As a tool to make things easier to read, you could use tables as in Table \ref{tab:my_table}\cite{SICD}. You can refer to the appendix where you have done your calculations to back up that your end results are correct, and to keep the flow structure clean.
\begin{table}[htb]
    \centering
    \caption{The tables presents the standard enthalpy of formation and the standard entropy used to calculate the thermodynamics of the combustion of carbon.}
    \begin{tabular}{ccc} % all columns are centered. You can also use lll, rrr or any combination of the three
    \toprule
                    & $\Delta_fH^{\circ}$ [\si{\kilo\joule\per\mole}]  & $S{^\circ}$ [\si{\joule\per\kelvin\per\mole}] \\
    \midrule
        \ce{C}      &  140 & 201 \\
        \ce{O2}     &  103 & 104 \\
        \ce{CO2}    &  430 & 210 \\
    \bottomrule
    \end{tabular}
    \label{tab:my_table}
\end{table}
In Table \ref{tab:my_table} all columns are centered. This is shown by the ccc tag inside the curly braces. You can also use lll, rrr, to create left aligned columns or right aligned columns respectively, or any combination of the three. The caption in tables should ALWAYS be on top of the actual table.

\FloatBarrier % Now the table doesn't flow over to any other sections